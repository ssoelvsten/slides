\documentclass[english, aspectratio=169]{beamer}
% english is for the language used in standard texts (figures, tables etc)
% aspectratio of 16:9 or set it for more old school to 4:3 (without the ':')

% ---------------------------------------------------------------------------- %
% Load base preamble
% ---------------------------------------------------------------------------- %
\usepackage{import}
\subimport{./preamble/}{beamer.tex}

% ---------------------------------------------------------------------------- %
% Local macros
% ---------------------------------------------------------------------------- %


% ---------------------------------------------------------------------------- %
% TITLEPAGE
% ---------------------------------------------------------------------------- %
\title{MLightDP}

\author{\textbf{Steffan Christ S\o lvsten} and Anna Mie Hansen}

\date{January 2020}

\institute{\includegraphics[width=0.2\linewidth]{./external/aulogo_uk_var2_black.eps}}

% ---------------------------------------------------------------------------- %
% Content
% ---------------------------------------------------------------------------- %
\begin{document}

\titleframe

\begin{frame}
  \begin{definition}[Differential Privacy]
    Given parameters $\epsilon \geq 0$ then a probabilistic algorithm $M : A
    \mapsto B$ is $\epsilon$-differentially private if

    \begin{align*}
      \forall a_1, a_2 &\in A\ \forall O \subset B :
      \\ & a_1 \Psi a_2 \implies
           \Pr \left[ M(a_1) \in O \right] \leq e^\epsilon \Pr \left[ M(a_2) \in O \right]
           \enspace .
    \end{align*}    
  \end{definition}
\end{frame}

\blankframe

\subsection{Randomness Alignment}
\begin{frame}
  \frametitle{Randomness Alignment}
  
  Consider $q^{(1)}, q^{(2)} \in \R$, where
  \begin{equation*}
    -1 \leq q^{(1)} - q^{(2)} = \hat{q} \leq 1 \enspace .
  \end{equation*}

  \pause
  
  If $\eta_1 \sim \mathit{Lap}(1/\epsilon)$ for $q^{(1)}$ we'd like a $\eta_2
  \sim \mathit{Lap}(1/\epsilon)$ for $q^{(2)}$ such that
  \begin{equation*}
    \eta_2 = f(\eta_1) = \eta_1 + \hat{q}
    \enspace ,
  \end{equation*}
  because then
  \begin{equation*}
    q^{(1)} + \eta_1 = q^{(1)} + \eta_1 + \hat{q} = q^{(2)} + \eta_2
    \enspace ,
  \end{equation*}

  \pause
  
  and then
  \begin{equation*}
    \Pr \left[ q^{(1)} + \mathit{Lap}(1/\epsilon) \in O \right]
    \leq e^\epsilon \Pr \left[ q^{(2)} + \mathit{Lap}(1/\epsilon)  \in O \right]
    \enspace .
  \end{equation*}
\end{frame}

\begin{frame}
  \frametitle{LightDP}
  Based on the above, Zhang and Kifer '17 propose a type system \emph{LightDP}
  with dependent types, to prove differential privacy of algorithms.

  \begin{equation*}
    \tau ::=
    \mathit{num}_{\mathbbm{d}}
    \ |\
    \mathit{num}_{*}
    \ |\
    \mathit{bool}_0
    \ |\
    \mathit{list}\ \tau
    \ |\
    \tau_1 \rightarrow \tau_2
  \end{equation*}
  
  The number type also contains the numerical \emph{distances} between the two
  executions.
\end{frame}

\begin{frame}
  \frametitle{LightDP}

  A program is differentially private, if

  \pause

  \begin{itemize}
  \item The returned value has distance $0$
  \end{itemize}
  
  \begin{prooftree}
    \AxiomC{$\Gamma \vdash e : \mathcal{B}_0$ or $\Gamma \vdash e : \mathit{list}\ \mathcal{B}_0$ or \dots}
    \RightLabel{(T-Return)\footnotemark[1]}
    \UnaryInfC{$\Gamma \vdash \mathit{return}\ e \rightharpoonup \mathit{return}\ e$}
  \end{prooftree}
  
  \vspace{1em}\pause

  \begin{itemize}
  \item The privacy budget variable $v_\epsilon$ is less or equal to $\epsilon$
    in the \emph{transformed} program on all return paths.
  \end{itemize}
  
  \begin{prooftree}
    \AxiomC{$\Gamma(\eta) = \mathit{num}_{\mathbbm{d}}$}
    \AxiomC{$e : \mathit{num}_0$}
    \RightLabel{(T-Laplace)\footnotemark[1]}
    \BinaryInfC{$\Gamma \vdash \eta :=\ \mathit{Lap}(e) \rightharpoonup
      \mathit{havoc}\ \eta;\ v_\epsilon := v_\epsilon + \lvert \mathbbm{d} \rvert / e$}
  \end{prooftree}

  \footnotetext[1]<2->{These rules are variations of the ones by Zhang and Kifer, to remove the disconnect
    between the rule and their/our implementation.}

\end{frame}

\begin{frame}[plain,noframenumbering]{}
  \pause
  \begin{center}
    \huge MLightDP
  \end{center}
\end{frame}

\begin{frame}[fragile]
  \frametitle{Example: \lstinline{laplace_mechanism.mldp}}
  
  Remember, the randomness alignment example from before where $-1 \leq q^{(1)}
  - q^{(2)} = \hat{q} \leq 1$ and adding $\eta \sim \mathit{Lap}(1/\epsilon)$ to
  $q$ makes them differentially private.

  \vspace{1em}

  \begin{onlyenv}<1-4>
    \begin{lstlisting}[language=none]
      *@\pause@*
      LaplaceMechanism(q: float)*@\pause@*
      {
        let eta : float      = Lap(1 / epsilon);*@\pause@*
        return q + eta;
      }
    \end{lstlisting}
  \end{onlyenv}
  \begin{onlyenv}<5>
    \begin{lstlisting}[language=none]
      precondition: -1.0 <= ^q /\ ^q <= 1.0
      LaplaceMechanism(q: float[^q], ^q : float[0])
      {
        let eta : float      = Lap(1 / epsilon);
        return q + eta;
      }
    \end{lstlisting}
  \end{onlyenv}
  \begin{onlyenv}<6>
    \begin{lstlisting}[language=none]
      precondition: -1.0 <= ^q /\ ^q <= 1.0
      LaplaceMechanism(q: float[^q], ^q : float[0])
      {
        let eta : float[-^q] = Lap(1 / epsilon);
        return q + eta;
      }
    \end{lstlisting}
  \end{onlyenv}
  \begin{onlyenv}<7>
    \begin{lstlisting}[language=none]
      precondition: -1.0 <= ^q /\ ^q <= 1.0
      LaplaceMechanism(q: float[*])
      {
        let eta : float[-^q] = Lap(1 / epsilon);
        return q + eta;
      }
    \end{lstlisting}
  \end{onlyenv}
\end{frame}

\begin{frame}[fragile]
  \frametitle{Example: \lstinline{laplace_mechanism.dfy}}

  The transformed \emph{Dafny} program then is:

  \vspace{1em}

  \begin{lstlisting}[language=none, basicstyle=\tiny]
    method LaplaceMechanism(epsilon : real, q : real, ghost q_hat0 : real)
      returns (_:real)
      requires epsilon > 0.0
      requires -1.0 <= q_hat0 && q_hat0 <= 1.0
    {*@\pause@*
      ghost var v_epsilon : real := 0.0;*@\pause@*

      assert 0.0 == 0.0; // T-Laplace
      assert 0.0 == 0.0; // T-OTimes
      assert 0.0 == 0.0; // T-OTimes*@\pause@*
      var eta :| 0.0 <= eta;
      if * { eta := eta; } else { eta := -eta; }*@\pause@*
      v_epsilon := v_epsilon + Abs(-q_hat0) / ((1 as real) / epsilon);*@\pause@*

      assert q_hat0 + -q_hat0 == 0.0; // T-Return
      assert v_epsilon <= epsilon; // T-Return*@\pause@*
      return q + eta;
    }
  \end{lstlisting}
\end{frame}

\begin{frame}
  \frametitle{Examples}
  
  \begin{table}
    \small
    \centering
    \begin{tabular}{l l}
      Verified?       & Example Program
      \\
      \checkmark      & \lstinline{laplace_mechanism.mldp}
      \\
      \checkmark      & \lstinline{sparse_vector.mldp} of Zhang and Kifer '17
      \\
      \checkmark \footnotemark[1]
                      & \lstinline{numerical_sparse_vector.mldp} of Zhang and Kifer '17
      \\
      \checkmark \footnotemark[1]
                      & \lstinline{gap_sparse_vector.mldp} of Wang et.\ al.\ '19
      \\
                      & \lstinline{partial_sum.mldp} of Zhang and Kifer '17
      \\
                      & \lstinline{smart_sum.mldp} of Zhang and Kifer '17
    \end{tabular}
  \end{table}

  \footnotetext[1]{We had to modify the given invariants to make these work.}
\end{frame}

\blankframe

\begin{frame}
  \frametitle{Comparison}
  
  \begin{table}
    \small
    \centering
    \begin{tabular}{c c l}
      \emph{LightDP} & \emph{MLightDP} & Feature
      \\ \hline \hline
          \checkmark & \checkmark      & DP Type checking
      \\ \hline
                     & \checkmark      & Output to a Software Verification tool
      \\ \hline
                     & \checkmark      & Semantic Analysis of distances
      \\ \hline
                     &                 & Refinement: Distance inference
      \\
                     & --              & Refinement: Mutable Dependency tracking
      \\ \hline
                     & --              & Free usage of the Laplace
      \\ \hline
                     &                 & Automating non-linearity
      \\ \hline
                     & --              & Other
    \end{tabular}
  \end{table}
\end{frame}

\begin{frame}
  \frametitle{MLightDP}

  The Goal of our project is to create a less prototypical experience than their
  current prototype.

  \begin{enumerate}
  \item Create more intuitive typing annotations.
  \item Be capable of accommodating the complete pipeline.
  \item Dispense with assumptions of Zhang and Kifer about the input.
  \item Make their rules of more fine-grained / less conservative.
  \item Soundly add new operators, functions, and alternative random distributions to
    the language.
  \end{enumerate}
\end{frame}

\end{document}
