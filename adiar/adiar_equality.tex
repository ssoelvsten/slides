\documentclass[english, aspectratio=169]{beamer}
% english is for the language used in standard texts (figures, tables etc)
% aspectratio of 16:9 or set it for more old school to 4:3 (without the ':')

% ---------------------------------------------------------------------------- %
% Load base preamble
% ---------------------------------------------------------------------------- %
\usepackage{import}
\subimport{../preamble/}{beamer.tex}

% ---------------------------------------------------------------------------- %
% Local settings
% ---------------------------------------------------------------------------- %
% https://tex.stackexchange.com/a/20613
\newcommand\hcancel[2][black]{\setbox0=\hbox{$#2$}%
  \rlap{\raisebox{.35\ht0}{\textcolor{#1}{\rule{\wd0}{1pt}}}}#2}

\newcommand{\B}[0]{\ensuremath{\mathbb{B}}}

\newcommand{\sort}[0]{\text{sort}}

\newcommand{\triple}[3]{\ensuremath{(#1, #2, #3)}}
\renewcommand{\arc}[3]{\ensuremath{#1 \xrightarrow{_{#2}} #3}}

% makes \addlegendimage available (typically only available within an
% axis environment):
\def\addlegendimage{\pgfplots@addlegendimage}
\makeatother

% ------------------------------------------------------------------------------
% TITLEPAGE
% ------------------------------------------------------------------------------
\title{
  Efficient Equality Checking\\for Non-Shared Binary Decision Diagrams
}

\author{\textbf{Steffan Christ S\o lvsten}, Jaco van de Pol}

\institute{\includegraphics[width=0.2\linewidth]{../external/aulogo_uk_var2_black.eps}}

\date{\today}

\begin{document}

\titleframe

% Problem:
% We do not have 'shared' BDDs in a single table
% Other packages use O(1) time and I/Os for equality checking

\blankframe

% Simple Solution 'f <-> g == T':
%   With Apply it is O(sort(N^2)). That's not good enough!

\begin{frame}
  \vspace{30pt}
  \begin{center}
    {\huge $f \leftrightarrow g \equiv \top$}
  \end{center}

  \pause
  \vspace{20pt}
  
  \begin{equation*}
    \underbrace{O(\sort(N^2))}_{\texttt{Apply}}
    + \underbrace{O(\sort(N^2))}_{\texttt{Reduce}}
    + \underbrace{O(1))}_{\text{check is } \top}
    = O(\sort(N^2))
  \end{equation*}
\end{frame}

% We need to do better!
%   Optimum would be 2N/B I/Os, i.e. only read each BDD once.

\blankframe

% Recall: Bryant's Theorem on canonicity
%   Isomorphism: ~
%
% Trivial negative cases

\begin{frame}
  \begin{theorem}[Bryant '86]
    Let $\pi$ be a variable order and $f : \B^n \rightarrow \B$ then there
    exists a unique (up to isomorphism) Reduced Ordered Binary Decision
    Diagram representing $f$ with ordering $\pi$.
  \end{theorem}

  \pause

  \vspace{10pt}
  {\bf Trivial cases: $f \not \equiv g$ if there is a mismatch in}

  \begin{tabular}{c l l l}
    {\tiny $\blacksquare$} & $N_f \neq N_g$       & Number of nodes            & $O(1)$ I/Os
    \\
    {\tiny $\blacksquare$} & $L_f \neq L_g$       & Number of levels           & $O(1)$ I/Os
    \\
    {\tiny $\blacksquare$} & $N_{f,i} \neq N_{g,i}$ & Number of nodes on a level & $O(L/B)$ I/Os
    \\
    {\tiny $\blacksquare$} & $L_{f,i} \neq L_{g,i}$ & Label of an $i$th level    & $O(L/B)$ I/Os
  \end{tabular}
    
\end{frame}

% O(sort(N)):
% Assume the isomorphism applies
% Cases we find a violation

\begin{frame}

  \begin{figure}
    \centering

    \begin{tikzpicture}[scale=1, every node/.style={transform shape}]
      % f
\node[shape = circle, black, draw = black] (i1)
{$x_i$};

\onslide<3->{
  \node[shape = rectangle, draw = black, below left=1.2cm and 0.4cm of i1] (child1_1)
  { $\top$ };
}
\onslide<1-2>{
  \node[shape = circle, draw = black, below left=1.2cm and 0.4cm of i1] (child1_1)
  { \phantom{$x_j$} };
}

\node[shape = circle, draw = black, below right=1.2cm and 0.4cm of i1] (child1_2)
{
  \only<1>{\phantom{$x_j$}}%
  \only<2->{$x_j$}%
};

\draw[->, dashed] (i1) edge (child1_1);
\draw[->] (i1) edge (child1_2);

% g
\node[shape = circle, black, draw = black, right=5cm of i1] (i2) {$x_i$};

\onslide<3->{
  \node[shape = rectangle, draw = black, below left=1.2cm and 0.4cm of i2] (child2_1)
  {
    \only<1-3,5->{$\top$}%
    \only<4>{$\bot$}%
  };
}
\onslide<1-2>{
  \node[shape = circle, draw = black, below left=1.2cm and 0.4cm of i2] (child2_1)
  { \phantom{$x_j$}};
}

\onslide<6->{
  \node[shape = rectangle, draw = black, below right=1.2cm and 0.4cm of i2] (child2_2)
  { $\top$ };
}
\onslide<1-5>{
\node[shape = circle, draw = black, below right=1.2cm and 0.4cm of i2] (child2_2)
{
  \only<1>{\phantom{$x_j$}}%
  \only<2-4>{$x_j$}%
  \only<5->{$x_k$}%
};
}

\draw[->, dashed] (i2) edge (child2_1);
\draw[->] (i2) edge (child2_2);

% isomorphism
\onslide<1-3>{
  \draw[blue, densely dashed, thick] (i1) edge (i2);
}

\node[shape = circle, black,below right=0.3cm and 2.3cm of i1] {$\Updownarrow$};

\onslide<1-3,5->{
  \draw[white!40!blue, densely dashed, thick](child1_1) edge[bend right=50] (child2_1);
}

\onslide<1-4>{
  \draw[black!40!blue, densely dashed, thick] (child1_2) edge[bend right=50] (child2_2);
}


    \end{tikzpicture}
  \end{figure}
  
\end{frame}

\begin{frame}

  \texttt{IsIsomorphic($f$, $g$)}
  \begin{itemize}
  \item Check whether root $v_f$ of $f$ and root $v_g$ of $g$ have a local violation.
  \item Check $\mathit{low}(v_f) \sim \mathit{low}(v_g)$ and $\mathit{high}(v_f)
    \sim \mathit{high}(v_g)$ ``recursively''.
  \end{itemize}
  Return \texttt{false} on first violation. If there are no violations then return \texttt{true}.

  \pause
  \begin{equation*}
    \underbrace{O(\sort(N^2))}_{\texttt{Apply}'}
    + \underbrace{\hcancel[gray]{O(\sort(N^2))}}_{\texttt{Reduce}}
    + \underbrace{\hcancel[gray]{O(1))}}_{\text{check is } \top}
    = O(\sort(N^2))
  \end{equation*}
  
\end{frame}

\begin{frame}
  \begin{figure}
    \centering

    \begin{tikzpicture}[scale=0.9, every node/.style={transform shape}]
      % f
\node[shape = circle, black, draw = black] (f1) {\tiny $x_i$};
\node[shape = circle, black, draw = black, right =of f1] (f2) {\tiny $x_i$};
\node[right =of f2] (f_dots) {$\dots$};
\node[shape = circle, black, draw = black, right =of f_dots] (fn) {\tiny $x_i$};

\draw [
thick,
decoration={
  brace,
  mirror,
  raise=0.5cm
},
decorate
] (-0.25,-0.1) -- ++(5.5,0)
node [pos=0.5,anchor=north,yshift=-0.55cm] {$N_{f,i}$};

% g

\node[shape = circle, black, draw = black, right =2cm of fn] (g1) {\tiny $x_i$};
\node[shape = circle, black, draw = black, right =of g1] (g2) {\tiny $x_i$};
\node[right =of g2] (g_dots) {$\dots$};
\node[shape = circle, black, draw = black, right =of g_dots] (gn) {\tiny $x_i$};

\draw [
thick,
decoration={
  brace,
  mirror,
  raise=0.5cm
},
decorate
] (7.3,-0.1) -- ++(5.5,0)
node [pos=0.5,anchor=north,yshift=-0.55cm] {$N_{g,i}$};

% isomorphism
\draw[blue, densely dashed]
  (f1) edge[bend left=20] (g2)
  (f2) edge[bend left=20] (g_dots)
  (f_dots) edge[bend left=20] (g1)
  (f_dots) edge[bend left=20] (gn)
  (fn) edge[bend left=20] (g_dots)
;
    \end{tikzpicture}
  \end{figure}

  \pause
  Return \texttt{false} if more than $N_{f,i} = N_{g,i}$ pairs of nodes $(v_f,
  v_g)$ are checked on level $i$.

  \begin{equation*}
    O(\sort(N))
  \end{equation*}
  
\end{frame}

% 2 N/B:
%   'Canonicity' of Adiar's BDDs

\blankframe

\begin{frame}
  % TODO: this slide is verbose!

  {\bf Observation}

  \vspace{-10pt}
  The output of \texttt{Reduce} has the following properties

  \begin{itemize}
  \item Nodes on level $i$ have their identifiers \emph{consecutively} numbered
    \begin{equation*}
      \mathit{MAX} - N_{f,i} + 1,
      \dots, \mathit{MAX-1}, \mathit{MAX}
      \enspace .
    \end{equation*}

  \item Nodes on level $i$ are output sorted by their children
    \begin{equation*}
      ((i_1, \mathit{id}_1), \mathit{low}_1, \mathit{high}_1)
      <_{\mathit{lex}(i, \mathit{low}, \mathit{high})}
      ((i_2, \mathit{id}_2), \mathit{low}_2, \mathit{high}_2)
      \enspace ,
    \end{equation*}
    where
    \begin{equation*}
      \forall (i,\mathit{id}) \ :\ (i,\mathit{id}) < \bot < \top
      \enspace
      \footnote{Assuming the BDD is not negated. If that is the case then $(i,\mathit{id}) < \top < \bot$.}
      .
    \end{equation*}
  \end{itemize}
 
\end{frame}

\begin{frame}

  \begin{theorem}
    If $G_f$ and $G_g$ are outputs of \texttt{Reduce}.
    \begin{center}
      $G_f \sim G_g$ $\iff$ For all $i \in [0; N)$ the node $G_f[i]$ matches
      $G_g[i]$ numerically.      
    \end{center}
  \end{theorem}
  \begin{proof}
    $\Leftarrow$ : Must describe the exact same graph.
    
    $\Rightarrow$ : Strong induction on BDD levels bottom-up \dots
  \end{proof}

  \pause
  \begin{corollary}
    If $G_f$ and $G_g$ are outputs of \texttt{Reduce} then $f \equiv g$ is
    computable using $2 \cdot N/B$ I/Os.%
    \footnote<2->{Assuming they are both unnegated (or both negated).}
  \end{corollary}
  
\end{frame}

\begin{frame}

  \begin{figure}
    \centering

    \begin{tikzpicture}[scale=0.9, every node/.style={transform shape}]
      % f
\node[shape = circle, black, draw = black] (f) {\scriptsize $(n, \_)$};

\node[shape = rectangle, draw = black, below left=1cm and 0cm of f] (f1) {$\bot$};
\node[shape = rectangle, draw = black, below right=1cm and 0cm of f] (f2) {$\top$};

\draw[->, dashed] (f) edge[bend right=10] (f1);
\draw[->] (f) edge[bend left=10] (f2);

% less-than
\node[shape = circle, black,right=2cm of f] {$<$};

% g
\node[shape = circle, black, draw = black, right=5cm of f] (g) {\scriptsize $(n, \_)$};

\node[shape = rectangle, draw = black, below left=1cm and 0cm of g] (g1) {$\top$};
\node[shape = rectangle, draw = black, below right=1cm and 0cm of g] (g2) {$\bot$};

\draw[->, dashed] (g) edge[bend right=10] (g1);
\draw[->] (g) edge[bend left=10] (g2);

    \end{tikzpicture}

    \caption{Base case: $n$}
  \end{figure}
  
\end{frame}

\begin{frame}
  
  \begin{figure}
    \centering

    \begin{tikzpicture}[scale=0.85, every node/.style={transform shape}]
      % f
\node[shape = circle, black, draw = black] (f1)
{\tiny $(i, \_)$};

\node[shape = circle, draw = black, below left=0.7cm and 0.2cm of f1] (f1_1) {};
\node[shape = circle, draw = black, below right=0.7cm and 0.2cm of f1] (f1_2) {};

\node[black, right =0.8cm of f1] (f1_less) {\small $<$};

\node[shape = circle, black, draw = black, right =2cm of f1] (f2)
{\tiny $(i, \_)$};

\node[shape = circle, draw = black, below left=0.7cm and 0.2cm of f2] (f2_1) {};
\node[shape = circle, draw = black, below right=0.7cm and 0.2cm of f2] (f2_2) {};

\node[black, right =0.8cm of f2] (f2_less) {\small $<$};

\node[shape = circle, black, draw = black, right =2cm of f2] (f3)
{\tiny $(i, \_)$};

\node[shape = circle, draw = black, below left=0.7cm and 0.2cm of f3] (f3_1) {};
\node[shape = circle, draw = black, below right=0.7cm and 0.2cm of f3] (f3_2) {};

\node[black, right =0.8cm of f3] (f3_less) {\small $<$};

\node[right =1.5cm of f3] (f_dots) {$\dots$};

\node[black, right =0.2cm of f_dots] (fdots_less) {\small $<$};

\node[shape = circle, black, draw = black, right =2cm of f_dots] (fn)
{\tiny $(i, \_)$};

\node[shape = circle, draw = black, below left=0.7cm and 0.2cm of fn] (fn_1) {};
\node[shape = circle, draw = black, below right=0.7cm and 0.2cm of fn] (fn_2) {};

\draw[->, dashed]
  (f1) edge (f1_1)
  (f2) edge (f2_1)
  (f3) edge (f3_1)
  (fn) edge (fn_1)
;
\draw[->]
  (f1) edge (f1_2)
  (f2) edge (f2_2)
  (f3) edge (f3_2)
  (fn) edge (fn_2)
;

\node[black, left=of f1] (f_label) {$f$:};

% g

\node[shape = circle, black, draw = black, below=3cm of f1] (g1)
{\tiny $(i, \_)$};

\node[shape = circle, draw = black, below left=0.7cm and 0.2cm of g1] (g1_1) {};
\node[shape = circle, draw = black, below right=0.7cm and 0.2cm of g1] (g1_2) {};

\onslide<2->{
  \node[black, right =0.8cm of g1] (g1_less) {\small $\stackrel{?}{<}$};
}

\node[shape = circle, black, draw = black, right =2cm of g1] (g2)
{\tiny $(i, \_)$};

\node[shape = circle, draw = black, below left=0.7cm and 0.2cm of g2] (g2_1) {};
\node[shape = circle, draw = black, below right=0.7cm and 0.2cm of g2] (g2_2) {};

\onslide<2->{
  \node[black, right =0.8cm of g2] (g2_less) {\small $\stackrel{?}{<}$};
}

\node[shape = circle, black, draw = black, right =2cm of g2] (g3)
{\tiny $(i, \_)$};

\node[shape = circle, draw = black, below left=0.7cm and 0.2cm of g3] (g3_1) {};
\node[shape = circle, draw = black, below right=0.7cm and 0.2cm of g3] (g3_2) {};

\onslide<2->{
  \node[black, right =0.8cm of g3] (g3_less) {\small $\stackrel{?}{<}$};
}

\node[right=1.5cm of g3] (g_dots) {$\stackrel{\tiny \phantom{?}}{\dots}$};

\onslide<2->{
  \node[black, right =0.2cm of g_dots] (fdots_less) {\small $\stackrel{?}{<}$};
}

\node[shape = circle, black, draw = black, right =2cm of g_dots] (gn)
{\tiny $(i, \_)$};

\node[shape = circle, draw = black, below left=0.7cm and 0.2cm of gn] (gn_1) {};
\node[shape = circle, draw = black, below right=0.7cm and 0.2cm of gn] (gn_2) {};

\draw[->, dashed]
  (g1) edge (g1_1)
  (g2) edge (g2_1)
  (g3) edge (g3_1)
  (gn) edge (gn_1)
;
\draw[->]
  (g1) edge (g1_2)
  (g2) edge (g2_2)
  (g3) edge (g3_2)
  (gn) edge (gn_2)
;

\node[black, left=of g1] (f_label) {$g$:};

% isomorphism

\draw[blue, dashed, thick]
  (f1) edge (g1)
  (f2) edge (g2)
  (f3) edge (g3)
  (fn) edge (gn)
;
    \end{tikzpicture}

    \caption{Induction case: $i < n$}
  \end{figure}
  
\end{frame}

% Algoritms in practice

\blankframe

\begin{frame}

  \begin{table}
    \centering
    \begin{tabular}{r | l}
      Algorithm                         & Time (s)
      \\ \hline
      $f \leftrightarrow g \equiv \top$ & 0.38
      \\
      $O(\sort(N))$                     & 0.058
      \\
      $2 \cdot N/B$                     & 0.006
    \end{tabular}

    \caption{Checking the (EPFL Benchmark) \emph{voter} circuit's single output
      gate ($\abs{N_f} = \abs{N_g} = 5.76~\text{MiB}$).}
  \end{table}
  
\end{frame}

\end{document}

%%% Local Variables:
%%% mode: latex
%%% TeX-master: t
%%% End:
