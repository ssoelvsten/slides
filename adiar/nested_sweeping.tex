\documentclass[english, aspectratio=169]{beamer}
% english is for the language used in standard texts (figures, tables etc)
% aspectratio of 16:9 or set it for more old school to 4:3 (without the ':')

% ---------------------------------------------------------------------------- %
% Load base preamble
% ---------------------------------------------------------------------------- %
\usepackage{import}
\subimport{../preamble/}{beamer.tex}

% ---------------------------------------------------------------------------- %
% Local settings
% ---------------------------------------------------------------------------- %
\newcommand{\B}[0]{\ensuremath{\mathbb{B}}}

\newcommand{\sort}[0]{\text{sort}}

\newcommand{\triple}[3]{\ensuremath{(#1, #2, #3)}}
\renewcommand{\arc}[3]{\ensuremath{#1 \xrightarrow{_{#2}} #3}}


\tikzstyle{plot_adiar}=[color=black, mark=o, mark size=1pt, line width=0.7pt]
\tikzstyle{plot_buddy}=[color=red, mark=o, mark size=1pt, line width=0.7pt]
\tikzstyle{plot_cudd}=[color=blue, mark=diamond, mark size=1pt, line width=0.7pt]
\tikzstyle{plot_sylvan}=[color=purple, mark=square, mark size=1pt, line width=0.7pt]

% Horizontal legends: https://tex.stackexchange.com/a/101578
% argument #1: any options
\makeatletter
\newenvironment{customlegend}[1][]{%
  \begingroup
  % inits/clears the lists (which might be populated from previous
  % axes):
  \pgfplots@init@cleared@structures
  \pgfplotsset{#1}%
}{%
  % draws the legend:
  \pgfplots@createlegend
  \endgroup
}%

% makes \addlegendimage available (typically only available within an
% axis environment):
\def\addlegendimage{\pgfplots@addlegendimage}
\makeatother

% ------------------------------------------------------------------------------
% 
% ------------------------------------------------------------------------------
% Opener: ...
% 
% Key points:
% Existential
% Basics (single variable)
% Push
% Bounce
% Relabelling
% Bounce
% 
% Take Home Message: How to do I/O-efficient "multi-recursion" (for RelProd)

% ------------------------------------------------------------------------------
% TITLEPAGE
% ------------------------------------------------------------------------------
\title{An External Memory Relational Product}

\author{\textbf{Steffan Christ S\o lvsten}, Jaco van de Pol}

\institute{\includegraphics[width=0.2\linewidth]{../external/aulogo_uk_var2_black.eps}}

\date{\today}

\begin{document}
\titleframe

\begin{frame}
  \begin{center}
    \LARGE

    $\mathit{RelProd}(S,T)
    \equiv
    (
    \ {\only<2>{\color{orange}} \exists \vec{x} .}\
    S(\vec{x}) \wedge T(\vec{x}, \vec{x'})
    \ ) [\vec{x'} / \vec{x}]$
  \end{center}
\end{frame}

\begin{frame}
  \frametitle{$\exists x_i .\ \phi \equiv \phi[x_i / \top] \vee \phi[x_i / \bot]$}

  \input{anim/relprod_old.tex}
\end{frame}

\blankframe

\begin{frame}
  \frametitle{$\exists \vec{x}.\ \phi(\dots)$ \quad (Push)}

  \input{anim/relprod_push.tex}
\end{frame}

\blankframe

\begin{frame}
  \frametitle{$\exists \vec{x}.\ \phi(\dots)$ \quad (Bounce)}

  \input{anim/relprod_bounce.tex}
\end{frame}

\blankframe

\begin{frame}
  \begin{center}
    \LARGE

    $\mathit{RelProd}(S,T)
    \equiv
    {\color{orange} (}
    \ \exists \vec{x} .\
    S(\vec{x}) \wedge T(\vec{x}, \vec{x'})
    \ {\color{orange} ) [\vec{x'} / \vec{x}]}$
  \end{center}
\end{frame}

\begin{frame}
  \begin{definition}
    A relabelling $\pi$ is monotonic if $x_i < x_j \implies \pi(x_i) < \pi(x_j)$
  \end{definition}

  \pause
  {\bf If $\pi$ is monotonic}

  \begin{itemize}
  \item \emph{1-Var} / \emph{Push}:

    Apply $\pi$ in $O(L_N)$ extra time during the final bottom-up Reduce
    sweep.

  \item \emph{Bounce}:

    $x_i' < x_j' \implies x_i < x_j$: $\pi$ can be applied during the
    outermost Reduce sweep.
  \end{itemize}

  \quad That is, applying $\pi$ can (essentially) be done for free.

  \pause
  {\bf If $\pi$ is not monotonic}

  \quad to be continued...
\end{frame}
\end{document}
